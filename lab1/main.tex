%----------------------------------------------------------------------------------------
%	
%----------------------------------------------------------------------------------------

%----------------------------------------------------------------------------------------
%	CONFIGURACION DOCUMENTO
%----------------------------------------------------------------------------------------

\documentclass{article}

\input{estructura.tex}

%----------------------------------------------------------------------------------------
%	INFORMACION
%----------------------------------------------------------------------------------------

\title{\textbf{Laboratorio 1: }Administrador de Procesos --- Parte A\@: Procesos y Comunicación Entre Procesos}

\author{
Facundo Nicolás Farias Lozano,
Juan Cruz Paez,
Tomás Agustín Muñoz
}

\date{\footnotesize\textsuperscript{\textbf{1}}Sistemas Operativos\\ \textsuperscript{\textbf{2}}Universidad Nacional de San Luis\\ \textsuperscript{\textbf{3}}2025}

%\date{Universidad Nacional de San Luis --- 2025}

%----------------------------------------------------------------------------------------

\begin{document}

\maketitle

%----------------------------------------------------------------------------------------
%	RESOLUCIÓN
%----------------------------------------------------------------------------------------

\section*{Ejercicio 1:}

\begin{figure}[h]
  \centering
  \includegraphics[width=0.85\textwidth]{resources/ej1a.png}
  \caption{System info}
\end{figure}

\subsection*{a- ¿Qué Sistema Operativo tiene instalado? ¿Qué distribución? ¿Qué versión?}
El sistema operativo que tengo instalado es Linux Mint 22.1 Cinnamon y es la version 6.4.8

\subsection*{b- ¿Cuántos procesadores posee su sistema de computadora? ¿Cuáles son sus características?}
En mi computadora tengo instalado un AMD Ryzen 5 5600X y cuenta con 12 procesadores(Hilos log)

\subsection*{c- ¿Cuál es la capacidad de memoria disponible?}

La memoria disponible es de \(15.5 GiB\).

\subsection*{d- ¿Qué placa de video o gráfica posee?}

La placa de vídeo es la GeForce GTX 1650 del fabricante NVIDIA.

\subsection*{e- ¿Cuál es la capacidad de disco que posee?}
La capacidad de disco con la que cuenta la computadora es de 1007.7Gb

%----------------------------------------------------------------------------------------

\section*{Ejercicio 2:}

\begin{figure}[h]
  \centering
  \includegraphics[width=0.85\textwidth]{resources/2.png}
  \caption{System monitor}
\end{figure}

\subsection*{a- ¿Qué información del sistema muestra?}

Al abrir el \textit{monitor de sistema} podemos observar:

\begin{itemize}
    \item \textbf{Uso de la CPU: } Muestra un histograma del uso de los procesador en tiempo real.
    \item \textbf{Uso de la memoria y espacio intercambiado (swap)\footnote{Herramienta de gestión de memoria que actua cuando la memoria RAM física se llena}: }
    Muestra de forma gráfica el uso de la memoria RAM y el espacio de intercambio.
    \item \textbf{Actividad de Red: } Muestra de manera gráfica la transmisión de  datos de red.
\end{itemize}

\subsection*{b- Mencione y describa qué información relevante sobre “Procesos” se puede mostrar (pestaña de “Procesos”).}

\begin{figure}[h]
  \centering
  \includegraphics[width=0.85\textwidth]{resources/ej2b2.png}
  \caption{System monitor: Process}
\end{figure}

\begin{figure}[h]
  \centering
  \includegraphics[width=0.5\textwidth]{resources/ej2b.png}
  \caption{System monitor: informacion de procesos}
\end{figure}

En dicha pestaña se muestra informacion en tiempo real de todos los procesos activos en el sistema. De ellos podemos ver:
\begin{itemize}
  \item \textbf{Nombre de Proceso: }Indica el programa que esta ejecutando
  \item \textbf{Usuario:} Indica que usuario inicio el proceso.
  \item \textbf{Uso de CPU:} Porcentaje del procesador que esta consumiendo el proceso
  \item \textbf{Id:} Numero unico que identifica a cada proceso
  \item \textbf{Memoria:} Cantidad de memoria RAM que esta consumiendo el proceso
  \item \textbf{Lectura de disco:} Cantidad de informacion leida desde el disco por el proceso
  \item \textbf{Escritura de disco:} Cantidad de informacion escrita en el disco por el proceso
  \item \textbf{Prioridad:} Indica que priorida le da el sistema al proceso frente a otros.
  \item \textbf{Estado:} Indica si el proceso esta durmiendo, ejecutandose,etc
\end{itemize}
  
\subsection*{c- ¿Qué operaciones se permite hacer respecto a los procesos?}
Las operaciones que se permiten hacer sobre un proceso son:
\begin{itemize}
  \item \textit{Open Files}
  \item \textit{Change Priority}
  \item \textit{Set Affinity}
  \item \textit{Stop}
  \item \textit{continue}
  \item \textit{Terminate}
  \item \textit{Memory Maps}
  \item \textit{Kill}
  \item \textit{End Procesos}
  \item \textit{Show Process Properties}
\end{itemize}

\subsection*{d- ¿Cómo se muestran las prioridades de un proceso?}

Las prioridades se muestran en una columna dedicada, donde cada proceso (fila) puede tomar los valores de: \textit{Very High}, \textit{High}, \textit{Normal}, 
\textit{Low}, \textit{Very Low}, \textit{Custom}

\subsection*{e- Inicializar las siguientes aplicaciones: un navegador Web, un procesador de texto y una
terminal/consola, luego responda observando la pestaña “Procesos” del Monitor de sistema:}

\begin{itemize}
    \item i- ¿Cuál es el ID (o PID, identificador del proceso) de cada proceso?
    \item ii- ¿En qué estado se encuentran cada uno de los procesos asociados a dichas aplicaciones?
    \item iii- ¿Qué proceso está en estado de ejecución?
    \item iv- Observar el tipo de cola (o Canal en espera) en la que puede estar un proceso según su
        estado.
    \item v- ¿Qué ocurre cuando se selecciona, para un proceso determinado, la opción de Detener,
        Finalizar o Matar un proceso? (presione botón derecho sobre un proceso elegido)
\end{itemize}

\begin{itemize}
  \item Brave web browser: Cuando se abre Brave muchos subprocesos de brave son abiertos (correspondiente a cada pestaña)
  \begin{itemize}
    \item ID 11249
    \item Estado: Alternan entre los procesos como running y sleeping, por lo que se encuentran algunos procesos en ejecución.
    \item Al detener el proceso se cierra completamente.
    \item Al matar el proceso también se cierra.
    \item El detener es similar.
  \end{itemize}
\end{itemize}
\begin{center}
    \begin{tabular}{|c|c|c|c|c|c|c|}   
    \hline
      \textbf{Proceso} & \textbf{ID} & \textbf{Estado} & \textbf{Ejecución?} & \textbf{Detención} & \textbf{Finalizar} & \textbf{Matar} \\
    \hline
      \textbf{Brave web browser} & 11249 & Running & Si & 34 & d & d\\
    \hline
  \end{tabular}
 \end{center}

 % \begin{center}
%     \begin{tabular}{|c|c|c|c|c|c|c|}   
%     \hline
%       \textbf{Proceso} & \textbf{ID} & \textbf{Estado} & \textbf{Ejecución?} & \textbf{Detención} & \textbf{Finalizar} & \textbf{Matar} \\
%     \hline
%       \textbf{Brave web browser} & 11249 & Running\footnote{En el momento de ejecutar Brave, se generan muchos procesos de Brave que intercalan entre running y sleeping} & Si & 34 & d & d\\

\subsection*{f- Desde la pestaña “Procesos”, ordene los procesos por número de proceso, donde se muestren los
siguientes datos (identificador del procesos, nombre del proceso, usuario, propietario, estado,
prioridad, memoria real):}

\begin{itemize}
    \item i. Muestre la información de los 10 primeros procesos (con captura de pantalla).
    \item ii. Complete la siguiente tabla de los procesos solicitados:
\end{itemize}

  Tabla de procesos de acuerdo al turnaround time
  \begin{center}
    \begin{tabular}{|c|c|c|c|c|}   
    \hline
      \textbf{Proceso} & \textbf{TT FIFO} & \textbf{TT SJF} & \textbf{TT Prio} & \textbf{TT RR} \\
    \hline
      \textbf{P1} & 10 & 34 & 22 & 34 \\
    \hline
  \end{tabular}
 \end{center}
%----------------------------------------------------------------------------------------

\section*{Ejercicio 3: Inicie el simulador Planificador de Procesos:}

\begin{itemize}
    \item En el sistema operativo Linux, descomprimir y abrir la carpeta del simulador.
    \item Una vez dentro de la carpeta del simulador “Planificador de Procesos”, abrir una “terminal o
consola” e ingresar: java VentanaPrincipal2
    \item Presione Enter para ejecutar el simulador.
    \item Lea el Manual de Ayuda del simulador
    \item Identifique los íconos del menú de herramientas, que representan las posibles operaciones del
simulador.
\end{itemize}

\begin{center}
    \begin{tabular}{|c|c|c|c|c|}   
    \hline
      \textbf{Proceso} & \textbf{TT FIFO} & \textbf{TT SJF} & \textbf{TT Prio} & \textbf{TT RR} \\
    \hline
      \textbf{P1} & 10 & 34 & 22 & 34 \\
    \hline
  \end{tabular}
 \end{center}


\subsection*{a- ¿Qué partes del diagrama del ciclo de vida de un proceso se pueden visualizar en la ventana
principal?}
\subsection*{b- ¿Qué información de los procesos puede ser visualizada? Ejemplifique.}
\subsection*{c- Respecto a la configuración de la simulación. ¿Qué información de los procesos se puede
configurar?}
\subsection*{d- ¿Qué información importante se puede observar una vez ejecutada la simulación?}
%----------------------------------------------------------------------------------------

\section*{Ejercicio 4: Inicialice la aplicación Terminal de Linux}


\subsection*{a- Utilizando el comando ps listar los procesos del sistema.}
\subsection*{b- Utilizando el comando ps con el parámetro -u listar los procesos del usuario actual únicamente.}
\subsection*{c- Utilizando el comando top listar los procesos.}
\subsection*{d- ¿Cuál es la diferencia de utilizar el comando top respecto de utilizar el comando ps?}
\subsection*{e- Observe la estructura jerárquica de los procesos en Linux utilizando el comando pstree.}
\subsection*{f- Ejecute el comando htop, luego visualice los estados de los procesos a través de la columna S (state).}
\subsection*{g- Utilizando el comando kill listar las opciones de parámetros posibles utilizando el parámetro -l, luego
investigue qué parámetros son necesarios para matar un proceso, muestre un ejemplo donde elige
un proceso para matar y luego lo mata aplicando el comando con los parámetros correspondiente.}

%----------------------------------------------------------------------------------------

\end{document}